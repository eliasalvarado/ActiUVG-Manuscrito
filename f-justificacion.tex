Tradicionalmente, la modalidad de gestión de proyectos de extensión dentro de la Universidad del Valle de Guatemala 
(UVG) suele depender de procesos manuales, comunicación dispersa (correos electrónicos, documentos físicos, 
promociones en redes sociales) y canales descoordinados. Provocando demoras, errores, pérdida de información, 
dificultad para el seguimiento de actividades y baja motivación por parte de los estudiantes. En consecuencia, 
la participación estudiantil se ve limitada, por ende, el impacto comunitario de estos proyectos de extensión se 
ve reducido. La modernización de estos procesos con soluciones tecnológicas que centralicen la gestión de los 
proyectos de extensión se ha vuelto una necesidad.

Un estudio reciente indica que las instituciones de educación superior deben de adaptar sus procesos, incluidos 
los de extensión, cuando operan bajo un contexto virtual, redefiniendo mecanismos de coordinación con el fin de 
asegurar inclusión, pertinencia social y calidad organizacional \cite{garcia2024gestion}.

Además, la consolidación de un sistema centralizado permite una gobernanza institucional más eficiente: facilitando 
la toma de decisiones informadas, promoviendo transparencia y, sobre todo, optimiza la gestión operativa \cite{barajas2024data}. 
Una plataforma web ayuda a reducir dependencias de múltiples sistemas heterogéneos. Lo que contribuye a que se 
agilicen los procesos de gestión.

La transformación digital en educación superior demuestra que adoptar ecosistemas digitales bien diseñados, 
trae beneficios tanto para los administradores como para los estudiantes. Los procesos internos son más eficientes, 
los costos operativos se reducen, y se mejora la accesibilidad y usabilidad para los usuarios finales \cite{pagerDuty2025digitalTransformation}. 
Una plataforma web como la propuesta facilitaría a directores un control centralizado, a organizaciones externas 
visibilidad institucional, y a estudiantes el acceso a oportunidades \cite{urmeneta2025design}.

La relevancia de modernizar, unificar y profesionalizar la gestión de los proyectos de extensión en la UVG la 
convierte en una necesidad. Se lograría mejorar la eficiencia administrativa, aumentar la motivación y participación 
estudiantil, y promover la transparencia en todo el proceso para todos los actores. Diversos estudios, como \cite{PINHO201880}, 
o artículos relevantes como \cite{barajas2024data}, \cite{pagerDuty2025digitalTransformation} y \cite{harper2021digitalTransformation} 
respaldan que portales web sirven como canales de comunicación entre 
todo el personal académico, siendo herramientas de apoyo en la gestión de procesos académicos y administrativos; 
además, la descentralización tiene un impacto negativo en la asignación de recursos, gobernanza de datos, y 
retención y éxito de los estudiantes. 
