\section{Extensión Universitaria}
La extensión universitaria es un componente fundamental de la misión en las universidades, pues busca llevar
el conocimiento académico y científico de su comunidad académica hacia la sociedad en general. 
Esta se puede realizar en diversas formas, tales como: conferencias, cursos cortos, talleres, proyectos de servicio 
comunitario, consultorías, entre otros. Cada una de estas actividades comparten el conocimiento y recursos adquiridos
en la Universidad con la sociedad, contrubuyendo así al desarrollo social, cultural, económico y científico \cite{bastias2004extension}. 

Además, la extensión universitaria representa un compromiso de las universidades con la sociedad, apoyando el fácil
acceso al conocimiento y fomentando la participación activa con el desarrollo integral de las comunidades. Esta actividad
ha sido fundamental en distintas instituciones educativas, promoviendo una educación inclusiva y relevante para las 
distintas necesidades sociales \cite{colotta2019politicasUniversitarias}.

Bajo este contexto, el Artículo 2 del Reglamento de Extensión de la Universidad del Valle se alinea con esta 
visión y compomiso, indicando lo siguiente: 
\begin{quote}
    \textit{"En la Universidad del Valle de Guatemala se define como Extensión al trabajo
    realizado en actividades orientadas al desarrollo social, cultural, económico y tecnológico de
    la comunidad."}
    \cite{reglamento2022uvg}
\end{quote}

    \subsection{Extensión Universitaria en el Desarrollo Social}
    El desarrollo social requiere que inicialmente exista un desarrollo personal de cada individuo
    que compone esa sociedad, pasando por un proceso de formación y adquisición o mejora de sus capacidades.
    La Universidad se considera como un agente clave en este proceso, un ente de transformación social. 
    Puesto que su comunidad académica serán los futuros profesionales, que en el desarrollo de su trabajo profesional,
    tendran la capacidad de influir directa o indirectamente en su entorno social 
    \cite{loor2022universidadSociedad,cedeña2012extensionDesarrollo}.

La Educación y la Universidad como institución educativa, tiene un papel fundamental en el desarrollo social,
pues su labor de gestión del conocimiento y formación de profesionales, contribuye al desarrollo de capacidades,
destrezas y habilidades, que permiten a los individuos participar activamente en la solución de problemáticas existentes
en su entorno social \cite{loor2022universidadSociedad}. Debido a esto, la extensión universitaria siendo pieza esencial,
expresa un compromiso con la sociedad, siendo un sistema abierto y flexible que atienda necesidades de la 
comunidad \cite{cedeña2012extensionDesarrollo}.

\section{Transformación Digital en la Educación Superior}
    \subsection{Era Digital}
    La era digital se caracteriza por la integración de tecnologías digitales en todos los aspectoss de la vida. Donde
    predominan los servicios y la \textit{\guillemetleft{}experiencia de usuario\guillemetright{}} sobre la producción de
    bienes. Dichos servicios incorporan \textit{\guillemetleft{}inteligencia\guillemetright{}} en los productos y 
    servicios digitales, rediseñando procesos y en varias ocasiones, se convierte la provisión de bienes en una prestación 
    de servicios\cite{chinkes2019transformacionDigital}.

    \subsection{Transformación Digital}
    Se refiere al proceso donde una entidad permite la integración de tecnologías digitales, por las cuales se puede dar
    respuestas estratégicas con valor agregado gracias a la innovación tecnológica. En el contexto universitario, la
    Transformación Digital impacta de lleno, puesto que genera una redefinición del modelo de Universidad 
    \cite{giusti2023transformacionDigital}.

    \subsection{Evolución de la Gestión Universitaria}
    Las unversidades, al igual que varias entidades, han tenido que adaptarse a los cambios tecnológicos y sociales que 
    la Era Digital ha traído. Esto ha llevado a una evolución en la gestión universitaria, favoreciendo naturalmente a 
    la eficiencia, transparencia y accesibilidad en los procesos administrativos y académicos. En su mayoría, esta 
    evolución ha sido impulsada por la propia necesidad de las unversidades de mantenerse relevantes, competitivas e 
    innovadoras frente a las demás instituciones como de su comunidad académica.Contemplando una sistematización de 
    procesos administrativos, académicos y técnicos. No solo haciendo referencia a la implementación de nuevos procesos, 
    sino a la innovación y mejoramiento de los procesos actuales, aún cuando estos ya sean digitales 
    \cite{chinkes2019transformacionDigital,cueva2020transformacionDigital}.
    
    De acuerdo a Chinkes y Julien, la transformación digital en instituciones de educación superior
    es una necesidad la cuál debe de ser abordada con una visión crítica y bajo las particularidades de cada institución
    \cite{chinkes2019transformacionDigital}.


    \subsection{Aplicación Web}
    Una aplicación Web es un programa informático o software el cuál se ejectua en Internet, sin la necesidad de que exista una instalación
    local en el dispositivo del usuario, el uso del navegador web basta para poder acceder a esta. Puede comprenderse como un conjunto de
    aplicaciones autónomas modulares que se pueden consumir desde una máquina a través de una red, como Internet.
    Estas aplicaciones permiten comunicación e intercambio de datos entre diferentes sistemas y plataformas, facilitando la interoperabilidad
    máquina a máquina sobre una red \cite{ibm2025WebServices}.
    Permitiendo acceso a la información de manera rápida y sencilla, así como realizar diversas interacciones que la aplicación Web permita 
    \cite{pardo2018comparacion}.

        \subsubsection{Beneficios de una aplicación web en procesos universitarios}
        Los beneficios de implementar una aplicación web en procesos universitarios son múltiples, entre los cuales se pueden destacar:
        \begin{itemize}
            \item \textbf{Ahorro de costos:} La migración a una aplicación web puede reducir costos operativos al tener menor 
            dependencia de recursos físicos.
            \item \textbf{Eficiencia operativa:} La agilización de procesos aumenta significativamente la eficiencia operativa, 
            reduciendo tiempos de espera y entregando información en tiempo real.
            \item \textbf{Innovación:} Frente a la competencia, tener un sistema digitalizado permite a las universidades demostrar 
            su dominio y adopción de tecnologías emergentes.
        \end{itemize}
        Según Urmeneta et al., la implementación de servicios web en instituciones educativas permite mejorar la eficiencia,
        accesibilidad y calidad de los servicios ofrecidos a estudiantes, docentes y personal administrativo. Abordando ineficiencias
        detectadas en los métodos manuales tradicionales, un sistema permite centralizar y agilizar el proceso de presentación, 
        revisión y aprobación de resultados. Además, ayuda a conectar y comunicar correctamente a todos los actores. \cite{urmeneta2025design}.

\section{Arquitectura de Aplicaciones Web}
La arquitectura de las aplicaciones web se refiere al modelo estructural que define la organización e interacción de los componentes de una
aplicación Web, tales como: servidores, bases de datos, lógica de negocio, interfaz de usuario, entre otros. Es de vital importancia, ya que 
la escalabilidad, rendimiento, y mantenibilidad de la aplicación Web dependen en gran medida de su arquitectura
\cite{romero2021arquitectura, universidadElBosque2024arquitectura}.

Generalmente, las aplicaciones Web se manejan bajo un modelo de cliente-servidor. El cual consta de los siguientes componentes clave:
\begin{itemize}
    \item \textbf{Capa de Presentación:} También conocido como Cliente o \textit{Frontend}, es la parte de la aplicación con la que el
            usuario interactura directamente. Esta incluye la interfaz de usuario y la experiencia de usuario (UI/UX), y su función 
            principal es presentar información y reunir datos de entrada.
    \item \textbf{Capa de Aplicación:} También conocido como Servidor o \textit{Backend}, es la parte de la aplicación que maneja la 
            lógica de negocio, procesamiento de datos, seguridad y comunicación con la base de datos.
    \item \textbf{Capa de Almacenamiento Persistente:} Esta capa tiene como función almacenar y gestionar los datos guardados en bases 
            de datos (SQL, NoSQL), garantizando una correcta persistencia de datos.
\end{itemize}

