\section{Extensión Universitaria}
La extensión universitaria es un componente fundamental de la misión en las universidades, pues busca llevar
el conocimiento académico y científico de su comunidad académica hacia la sociedad en general. 
Esta se puede realizar en diversas formas, tales como: conferencias, cursos cortos, talleres, proyectos de servicio 
comunitario, consultorías, entre otros. Cada una de estas actividades comparten el conocimiento y recursos adquiridos
en la Universidad con la sociedad, contrubuyendo así al desarrollo social, cultural, económico y científico
\cite{bastias2004extension}. 

Además, la extensión universitaria representa un compromiso de las universidades con la sociedad, apoyando el fácil
acceso al conocimiento y fomentando la participación activa con el desarrollo integral de las comunidades. Esta actividad
ha sido fundamental en distintas instituciones educativas, promoviendo una educación inclusiva y relevante para las 
distintas necesidades sociales \cite{colotta2019politicasUniversitarias}.

Bajo este contexto, el Artículo 2 del Reglamento de Extensión de la Universidad del Valle se alinea con esta 
visión y compomiso, indicando lo siguiente: 
\begin{quote}
    \textit{"En la Universidad del Valle de Guatemala se define como Extensión al trabajo
    realizado en actividades orientadas al desarrollo social, cultural, económico y tecnológico de
    la comunidad."}
    \cite{reglamento2022uvg}
\end{quote}

    \subsection{Extensión Universitaria en el Desarrollo Social}
    El desarrollo social requiere que inicialmente exista un desarrollo personal de cada individuo
    que compone esa sociedad, pasando por un proceso de formación y adquisición o mejora de sus capacidades.
    La Universidad se considera como un agente clave en este proceso, un ente de transformación social. 
    Puesto que su comunidad académica serán los futuros profesionales, que en el desarrollo de su trabajo profesional,
    tendran la capacidad de influir directa o indirectamente en su entorno social 
    \cite{loor2022universidadSociedad,cedeña2012extensionDesarrollo}.

La Educación y la Universidad como institución educativa, tiene un papel fundamental en el desarrollo social,
pues su labor de gestión del conocimiento y formación de profesionales, contribuye al desarrollo de capacidades,
destrezas y habilidades, que permiten a los individuos participar activamente en la solución de problemáticas existentes
en su entorno social \cite{loor2022universidadSociedad}. Debido a esto, la extensión universitaria siendo pieza esencial,
expresa un compromiso con la sociedad, siendo un sistema abierto y flexible que atienda necesidades de la 
comunidad \cite{cedeña2012extensionDesarrollo}.

\section{Transformación Digital en la Educación Superior}
    \subsection{Era Digital}
    La era digital se caracteriza por la integración de tecnologías digitales en todos los aspectoss de la vida. Donde
    predominan los servicios y la \textit{\guillemetleft{}experiencia de usuario\guillemetright{}} sobre la producción de
    bienes. Dichos servicios incorporan \textit{\guillemetleft{}inteligencia\guillemetright{}} en los productos y 
    servicios digitales, rediseñando procesos y en varias ocasiones, se convierte la provisión de bienes en una prestación 
    de servicios\cite{chinkes2019transformacionDigital}.

    \subsection{Transformación Digital}
    Se refiere al proceso donde una entidad permite la integración de tecnologías digitales, por las cuales se puede dar
    respuestas estratégicas con valor agregado gracias a la innovación tecnológica. En el contexto universitario, la
    Transformación Digital impacta de lleno, puesto que genera una redefinición del modelo de Universidad 
    \cite{giusti2023transformacionDigital}.

    \subsection{Evolución de la Gestión Universitaria}
    Las unversidades, al igual que varias entidades, han tenido que adaptarse a los cambios tecnológicos y sociales que 
    la Era Digital ha traído. Esto ha llevado a una evolución en la gestión universitaria, favoreciendo naturalmente a 
    la eficiencia, transparencia y accesibilidad en los procesos administrativos y académicos. En su mayoría, esta 
    evolución ha sido impulsada por la propia necesidad de las unversidades de mantenerse relevantes, competitivas e 
    innovadoras frente a las demás instituciones como de su comunidad académica.Contemplando una sistematización de 
    procesos administrativos, académicos y técnicos. No solo haciendo referencia a la implementación de nuevos procesos, 
    sino a la innovación y mejoramiento de los procesos actuales, aún cuando estos ya sean digitales 
    \cite{chinkes2019transformacionDigital,cueva2020transformacionDigital}.
    
    De acuerdo a Chinkes y Julien, la transformación digital en instituciones de educación superior
    es una necesidad la cuál debe de ser abordada con una visión crítica y bajo las particularidades de cada institución
    \cite{chinkes2019transformacionDigital}.


    \subsection{Aplicación Web}
    Una aplicación Web es un programa informático o software el cuál se ejectua en Internet, sin la necesidad de que 
    exista una instalación local en el dispositivo del usuario, el uso del navegador web basta para poder acceder a esta.
    Puede comprenderse como un conjunto de aplicaciones autónomas modulares que se pueden consumir desde una máquina a 
    través de una red, como Internet. Estas aplicaciones permiten comunicación e intercambio de datos entre diferentes 
    sistemas y plataformas, facilitando la interoperabilidad máquina a máquina sobre una red \cite{ibm2025WebServices}.
    Permitiendo acceso a la información de manera rápida y sencilla, así como realizar diversas interacciones que la 
    aplicación Web permita \cite{pardo2018comparacion}.

        \subsubsection{Beneficios de una aplicación web en procesos universitarios}
        Los beneficios de implementar una aplicación web en procesos universitarios son múltiples, entre los cuales se 
        pueden destacar:
        \begin{itemize}
            \item \textbf{Ahorro de costos:} La migración a una aplicación web puede reducir costos operativos al tener 
            menor dependencia de recursos físicos.
            \item \textbf{Eficiencia operativa:} La agilización de procesos aumenta significativamente la eficiencia
            operativa, reduciendo tiempos de espera y entregando información en tiempo real.
            \item \textbf{Innovación:} Frente a la competencia, tener un sistema digitalizado permite a las universidades
            demostrar su dominio y adopción de tecnologías emergentes.
        \end{itemize}
        Según Urmeneta et al., la implementación de servicios web en instituciones educativas permite mejorar la eficiencia,
        accesibilidad y calidad de los servicios ofrecidos a estudiantes, docentes y personal administrativo. Abordando 
        ineficiencias detectadas en los métodos manuales tradicionales, un sistema permite centralizar y agilizar el proceso 
        de presentación, revisión y aprobación de resultados. Además, ayuda a conectar y comunicar correctamente a todos los 
        actores. \cite{urmeneta2025design}.

\section{Arquitectura de Aplicaciones Web}
La arquitectura de las aplicaciones web se refiere al modelo estructural que define la organización e interacción de los 
componentes de una aplicación Web, tales como: servidores, bases de datos, lógica de negocio, interfaz de usuario, entre 
otros. Es de vital importancia, ya que la escalabilidad, sostenibilidad, rendimiento, y mantenibilidad de la aplicación Web 
dependen en gran medida de su arquitectura \cite{romero2021arquitectura, universidadElBosque2024arquitectura}.

\subsection{\textit{Model-View-Controller} (MVC)}
El patrón de diseño Modelo-Vista-Controlador (MVC, por sus siglas en inglés) es una arquitectura que separa la aplicación 
en capas, dividiendo las aplicaciones en tres componentes lógicos \cite{RANA2022271, yalupin2021programacionFrontEndBackEnd}:
\begin{itemize}
    \item \textbf{Modelo:} También conocido como la capa de almacenamiento persistente, esta componente tiene como función 
            almacenar y gestionar los datos guardados en bases de datos (SQL, NoSQL), garantizando una correcta persistencia
            de datos.
    \item \textbf{Vista:} También conocido como la capa de presentación, es la parte de la aplicación con la que el
            usuario interactura directamente. Este componente incluye la interfaz de usuario y la experiencia de usuario 
            (UI/UX), y su función principal es presentar información y reunir datos de entrada.
    \item \textbf{Controlador:} También conocido como la capa de aplicación, es la parte de la aplicación que maneja la 
            lógica de negocio, procesamiento de datos, seguridad y comunicación con la base de datos.
\end{itemize}
Este patrón de diseño, generalmente utilizado en aplicaciones Web \textit{client-server} o sistemas distribuidos, ofrece muchas
ventajas, entre ellas la facilidad de mantenimienbo, escalabilidad y reutilización. Ya que cada componente se puede gestionar de
manera independiente \cite{RANA2022271}.

\section{\textit{Frontend}}
El término \textit{Frontend}, de la mano con el componente \textit{Vista}, hace referencia a la interfaz gráfica de usuario
(GUI, por sus siglas en inglés), la cual es la parte visual con la que el usuario interactúa directamente. En pocas palabras, 
es la parte que permite la interacción del usuario con la aplicación. Bajo el contexto Web, la GUI vista por el usuario desde su
navegador se le conoce como modelo de objetos del documento (DOM, por sus siglas en inglés) \cite{aws2025frontendBackend}.

\subsection{Seguridad}
La seguridad en el \textit{Frontend} se basa en estrategias de validación de entradas, desactivación de configuraciones
predeterminadas y la implementación de políticas de seguridad de contenido (CSP, por sus siglas en inglés)
\cite{aws2025frontendBackend}.

\subsection{Renderizado Web}
El renderizado Web es el proceso mediante el cual el navegador interpreta y transforma el código (HTML, CSS, JavaScript) 
en una página visual e interactiva para el usuario \cite{vercel2024renderingStrategy}.
Existen varias estrategias de renderizado web, entre las cuales se encuentran :

\subsubsection{\textit{Static Site Generation} (SSG)}
SSG es una técnica de renderizado donde se preprocesan las páginas en el momento de la compilación, dando como resultado archivos
HTML estáticos que permiten enviarlos de forma más rápida y eficiente. Ideal para páginas con contenido que cambie con muy poca 
frecuencia. 

\textbf{Beneficios:}
\begin{itemize}
    \item Excelente rendimiento SEO.
    \item Reducción de la carga del servidor.
    \item Carga de página rápida.
    \item Costes de infraestructura mínimos.
\end{itemize}

\textbf{Consideraciones:}
\begin{itemize}
    \item Páginas con gran número de páginas, aumentan el tiempo de compilación.
    \item Una actualización de contenido requiere una compilación completa.
\end{itemize}

\subsubsection{\textit{Incremental Static Regeneration} (ISR)}
ISR permite la actualización de páginas espefícicas sin necesidad de compilar todo de nuevo. Combinando los beneficios de SSG con
una actualización más fluida y rápida. Ideal cuando SSG no se da a basto debido a su tiempo de compilación.

\textbf{Beneficios:}
\begin{itemize}
    \item Carga de página rápida.
    \item Permite actualizaciones bajo demanda, sin la necesidad de compilar todo.
    \item Adaptación a gran número de páginas.
\end{itemize}

\textbf{Consideraciones:}
\begin{itemize}
    \item Contoles de caché más complejos.
\end{itemize}

\subsubsection{\textit{Server-Side Rendering} (SSR)}
SSR es una técnica de renderizado donde el servidor genera un HTML completo en cada petición y se lo envía al cliente tal cual, 
lo cuál permite contenido en tiempo real y personalizado. Ideal para aplicaciones con contenido dinámico y cambio constante.

\textbf{Beneficios:}
\begin{itemize}
    \item Contenido actualizado.
    \item Mejor rendimiento SEO.
\end{itemize}

\textbf{Consideraciones:}
\begin{itemize}
    \item Mayor carga en el servidor.
    \item Tiempos de carga inicial más lentos.
\end{itemize}

\subsubsection{\textit{Client-Side rendering} (CSR)}
CSR es una técnica donde el servidor genera un HTML básico, dejando que el cliente sea el encargado de renderizar el contenido
completo mediante JavaScript. Ideal para aplicaciones web interactivas y con alta dinámica en su contenido.

\textbf{Beneficios:}
\begin{itemize}
    \item Interacciones en tiempo real con datos externos.
    \item Reducción de la carga del servidor.
    \item Experiencia de usuario fluida.
\end{itemize}

\textbf{Consideraciones:}
\begin{itemize}
    \item Carga inicial más lenta.
    \item Se pierde rendimiento SEO.
    \item Gestión rigurosa del estado de la aplicación.
\end{itemize}

\subsubsection{\textit{Partial Prerendering} (PPR)}
PPR es una técnica que se encuentra en fase experimental, pero busca combinar los beneficios de las demás estrategias. Para ello, 
prerenderiza desde el servidor cualquier parte estática de la página y luego transmite el contenido dinámico basándose en los
límties de React Suspense. Al estar en fase experimental, su uso ideal aún no está definido, así como sus limitaciones.

\textbf{Beneficios:}
\begin{itemize}
    \item Carga inicial rápida (como SSG).
    \item Contenido dinámico actualizado (como SSR/CSR).
    \item Mejor rendimiento SEO.
    \item Reducción de gastos generales de desarrollo.
\end{itemize}

\cite{vercel2024renderingStrategy}

\section{\textit{Backend}}
El término \textit{Backend}, de la mano con el componente \textit{Controlador} y el componente \textit{Modelo}, es el
encargado de administrar la funcionalidad general de la aplicación. Esto incluye la lógica de negocio y la gestión del 
almacenamiento persistente de datos. Además, el \textit{Backend} es el responsable de manejar cualquier solicitud 
proveniente del \textit{Frontend}, donde la procesa y devuelve una respuesta \cite{aws2025frontendBackend}.

\subsection{Seguridad}
La seguridad en el \textit{Backend} se basa en la seguridad del almacenamiento de datos y el tránsito. Esto incluye
encriptación de datos, controles de acceso, seguridad de sesión y protección contra amenzas comunes tales como: 
inyección SQL, \textit{cross-site scripting} (XSS) y \textit{cross-site request forgery} (CSRF) \cite{aws2025frontendBackend}.

\section{Tecnologías de Implementación}

\subsection{\textit{Amazon Web Services} (AWS)}
AWS es una plataforma de \textit{cloud computing} más completa y adoptada a nivel mundial. Permite a empresas y desarrolladores
acceder a una amplia gama de servicios de computación. Entre los cuales destacan \cite{aws2026aws}:
\begin{itemize}
    \item Servicios de computación: Amazon Elastic Compute Cloud (EC2), AWS Lambda, Amazon Lightsail, entre otros. Que permiten
            ejecutar servidores virtuales, aplicaciones sin servidor, contenedores, etc.
    \item Almacenamiento: Amazon Simple Storage Service (S3), Amazon Elastic Block Store (EBS), Amazon Glacier, entre otros. Que
            permiten almacenar cualquier cantidad de datos, archivos u objetos de manera segura y escalable.
    \item Bases de datos: Amazon Relational Database Service (RDS), Amazon DynamoDB, Amazon Aurora, entre otros. Que permiten
            iniciar, gestionar y escalar bases de datos relacionales o no relacionales de manera sencilla y eficiente.
\end{itemize}

\subsection{Docker}
Docker es una plataforma de código abierto que permite crear, desplegar y ejecutar aplicaciones dentro de contenedores. Un
contenedor es una unidad de software ligera y portátil capaz de empaquetar código, librerías, dependencias y configuraciones
esenciales para ejecutar una aplicación. Por lo que, actúa como una máquina virtual virtualizada. Altamente utilizado en el 
desarrollo de aplicaciones modernas, incluyendo aplicaciones web, gracias a su gran control y consistencia en el entorno de 
ejecución, permitiendo un mayor control sobre la infraestructura y despliegues \cite{aws2025docker}.

\subsection{PostgreSQL}
PostgreSQL es un sistema de base de datos relacional de código abierto que amplía las capacidades de SQL tradicional.
Combinando numerosas funciones avanzadas que logran almacenar y escalar de forma segura las cargas de trabajo de datos más
complejas. Gracias a su arquitectura, fiabilidad, integridad, extensibilidad y gran comunidad, se ha convertido en una de las
bases de datos más utilizadas y sólidas. Cumpliendo con ACID (Atomicidad, Consistencia, Aislamiento y Durabilidad) y su alta
gama en tipos de datos, funcionalidades internas y extensiones, lo hacen ideal para la mayoría de aplicaciones, incluyendo
aplicaciones web \cite{postgresql2026about}.

\subsection{Node js}
Node js es un entorno de ejecución de código abierto para JavaScript que utiliza el motor V8 de Google Chrome, ideal para
construir aplicaciones web escalables, alto rendimiento, y lo más importante, no bloqueante. Gracias a que utiliza un modelo
basado en eventos de E/S (entrada/salida) lo que permite manejar múltiples conexiones simultáneamente sin bloquearse. 
Permitiendo así crear servidores web, APIs RESTful, aplicaciones en tiempo real (chat, juegos, etc.) \cite{cantelon2014node}.

\subsection{JavaScript}
JavaScript es un lenguaje de programación utilizado principalmente para el desarrollo de páginas web interactivas. Puede ser
utilizado tanto en el \textit{Frontend} como en el \textit{Backend}, gracias a entornos de ejecución como \textit{Node.js}.
Es un lenguaje interpretado, orientado a objetos y basado en prototipos y multiparadigma; incluso tiene acceso al
\textit{DOM} de la página web, lo cuál permite manipular elementos HTML \cite{aws2025javascript}.

\subsection{HTML}
El lenguaje de marcas de hipertexto (HTML, por sus siglas en inglés) es la base de la mayoría de las páginas web. Define la
estructura y contenido de una página web por medio de etiquetas y atributos que indican al navegador qué y cómo mostrar
\cite{aws2025javascript}.

\subsection{CSS}
Las hojas de estilo en cascada (CSS, por sus siglas en inglés) es un lenguaje de reglas de estilo que ayudan a definir estilos
al contenido HTML. Estas reglas indican al navegador cómo renderizar los elementos HTML \cite{aws2025javascript}.

\subsection{TypeScript}
TypeScript es un superconjunto de JavaScript, lo que significa que un código JavaScript válido también es un código TypeScript.
TypeScript perfecciona JavaScript al agregar tipos en la sintaxis, lo que permite a herramientas de edición de código detectar
errores en tiempo de desarrollo, también conocido como \textit{type-safe} \cite{aws2025javascript}.

\subsection{React}
React es una biblioteca JavaScript de código abierto creada por Facebook (Meta). Desde su lanzamiento en 2013, e ha convertido
en una de las bibliotecas más populares para el \textit{Frontend}, superando a otras tecnologías como Angular y Vue.js.
Gracias a su enfoque basado en componentes, React permite construir aplicaciones Web altamente dinámicas y escalables.
React crea un DOM virtual en cada componente (copia del DOM) para compararlo con el estado del DOM real y aplicar el cambio 
solamente al elemento actualizado, mejorando significativamente el rendimiento de la aplicación al no tener que renderizar 
toda la página nuevamente. Cada componente es una fusión de la estructura HTML y la lógica JavaScript, lo que se conoce como
sintaxis JSX (JavaScript XML) \cite{saavedra2023react}.

\subsubsection{Beneficios}
\begin{itemize}
    \item El DOM virtual ayuda a ahorrar recursos y tráfico, mejorando el rendimiento.
    \item La reutilización de componentes permite un desarrollo más limpio y estructurado.
    \item Su estructura basada en comopnentes ayuda a la hora de escalar o darle mantenimiento a la aplicación.
    \item Integración sencilla con otras bibliotecas.
\end{itemize}

\subsubsection{Consideraciones}
\begin{itemize}
    \item La biblioteca en sí puede aumentar el tamaño total de la aplicación.
    \item Evoluciona constantemente, por lo que requiere aprendizaje continuo.
    \item Al no ser un \textit{Framework}, requiere la integración de otras bibliotecas para funcionalidades adicionales.
\end{itemize}

\subsection{TanStack}
TanStack es una colección de bibliotecas de código abierto de alto rendimiento para construir aplicaciones web modernas.
Gracias a su versatilidad, puede integrarse a un proyecto sin importar el \textit{framework}, pero es comúnmente utilizado
con React, Vue, Solid y Angular; ya que sus paquetes son especialmente para el \textit{Frontend} \cite{rauhala2025comparison}.

\subsubsection{TanStack Router}
TanStack Router es una biblioteca de enrutamiento \textit{type-safe} basado en el sistema de archivos. Compartiendo muchas 
características con React Router, pero con soporte SSR, admitiendo múltiples tipos de rutas diferentes. Cada ruta puede 
tener sus propios \textit{loaders}, \textit{actions} y \textit{error boundaries} \cite{rauhala2025comparison}.

\subsubsection{TanStack Query}
TanStack Query, antes conocido como React Query, es una biblioteca de gestión de estado asíncrono \textit{type-safe} para 
\textit{fetching}, almacenamiento en caché, sincronización y actualización de datos del servidor. Se basa en tres conceptos
principales \cite{garg2025react}:
\begin{itemize}
    \item \textbf{\textit{Queries}:} Para \textit{fetching} y lectura de datos.
    \item \textbf{\textit{Mutations}:} Para crear, actualizar o eliminar datos.
    \item \textbf{\textit{Query Client}:} Punto central de coordinación.
\end{itemize}
Además, ofrece características como TypeScript integrado, soporte para \textit{pagination} e \textit{infinite scrolling} 
(altamente optimizados) y utilidades \textit{server-state} \cite{garg2025react}.

\subsubsection{TanStack Form}
TanStack Form es una biblioteca para la gestión de estado de formularios \textit{type-safe} de alto rendimiento. Diseñada
para ser flexible y adaptable a cualquier caso de uso, permite la creación de formularios complejos con validaciones
personalizadas, así como el manejo de errores. Además, permite integrar fácilmente cualquier componente de interfaz de
usuario propia o de terceros \cite{tanstackSFform}.

\subsection{\textit{Material UI} (MUI)}
MUI es una biblioteca de componentes de interfaz de usuario de código abierto para React, diseñada bajo principios de
\textit{Material Design} de Google. Esta biblioteca proporciona una gran variedad de componentes preconstruidos
personalizables y adaptables en cualquier aspecto gracias a su robusto sistema de temas. Compatible con TypeScript,
hace un desarrollo más fluido y seguro \cite{mui2025materialUI}.

\subsection{Axios}
Axios es un Cliente HTTP basado en promesas para el navegador y Node JS. Gracias a que es isomórfico, puede ejecutarse tanto
en el \textit{Frontend} como en el \textit{Backend}. Una de sus grandes ventajas es que transforma de manera automática las
respuestas del servidor a formato JSON, también que permite interceptar peticiones y/o respuestas para realizar acciones
personalizadas, además soporta \textit{timeouts}, cancelación de peticiones y manejo de errores \cite{axios2025axios}.

\subsection{Zustand}
Zustand es una biblioteca ligera de gestión de estado para React. Funciona mediante \textit{hooks} los cuales permiten crear
estados globales sin la necesidad de componentes proveedores de contexto. Busca minimizar tanto el código como la complejidad,
ofreciendo estados inmutables y actualizaciones predecibles con un rendimiento óptimo. Además, es compatible con TypeScript
\cite{zustand2025}.

\subsection{\textit{JSON Web Token} (JWT)}
\textit{JSON Web Token} es un estándar abierto (RFC 7519) el cuál ayuda a definir de manera compacta y autónoma una forma de
transmitir información de manera segura entre partes como un JavaScript Object Notation (JSON, por sus siglas en inglés). Siendo
segura al contener una firma digital, lo que permite verificar la integridad de su contenido, esta firma puede ser creada
utilizando un secreto o un par de claves pública/privada utilizando RSA o ECDSA. En general, un JWT se utiliza para la
autenticación y autorización dentro de una aplicación. Cada solicitud que un usuario logueado realice, llevará su JWT,
permitiendo así acceder a los recursos y acciones protegidas por la aplicación \cite{auth02024jwt}.

En su forma compacta, un JWT se compone de tres partes codificadas en Base64Url separadas por puntos (.), las cuales son:
\begin{itemize}
    \item \textit{Header}: Suele constar de dos partes, el tipo de token (JWT) y el algoritmo de firma utilizado, como HMAC SHA256 o RSA.
    \item \textit{Payload}: Contiene las reclamaciones, son declaraciones sobre una entidad (el usuario) e información adicional.
            Esta información puede ser datos del usuario, permisos, entre otros.
    \item \textit{Signature}: La firma se crea tomando el contenido codificado del \textit{Header} y el \textit{Payload} y un
            secreto, utilizando el algoritmo especificado en el \textit{Header}. Esta firma garantiza que el JWT no ha sido alterado
            después de su emisión.
\end{itemize}

\subsection{\textit{Application Programming Interface} (API)}
Una interfaz de programa de aplicación (API, por sus siglas en inglés) define las reglas a seguir para la comunicación entre
sistemas de software, a este conjunto de reglas se le conoce como \textit{endpoints}. En pocas palabras, una API es un
conjunto de definiciones que ayudan a estandarizar la comunicación con cierto sistema (servidor), permitiendo que distintas 
aplicaciones (cliente) puedan interactuar con dicho sistema sin la necesidad de conocer su implementación interna 
\cite{aws2025restfulAPI}.

\subsection{\textit{Representational State Transfer} (REST)}
La transferencia de estado representacional (REST, por sus siglas en inglés) es una arquitectura de software que define un
conjunto de condiciones como intefaz uniforme, operaciones HTTP estándar, ausencia de estado y representaciones intercambiables
de información. Una API que cumple con estas condiciones se le conoce como
API RESTful \cite{aws2025restfulAPI}.

\subsubsection{Solicitud}
Para que una API RESTful responda correctamente hacia una solicitud HTTP, la solicitud debe de contener los siguientes
componentes principales \cite{aws2025restfulAPI}:

\begin{itemize}
    \item \textbf{\textit{Endpoint}:} Generalmente, es una URL (siglas en inglés para Localizador Uniforme de Recursos) que
            identifica un recurso único en el servidor.
    \item \textbf{Método:} Define la acción que se desea realizar sobre el recurso. Los métodos HTTP más comunes son:
            \begin{itemize}
                \item \textbf{GET:} Recuperar datos de un recurso.
                \item \textbf{POST:} Crear un nuevo recurso.
                \item \textbf{PUT:} Actualizar un recurso existente.
                \item \textbf{DELETE:} Eliminar un recurso.
            \end{itemize}
    \item \textbf{Encabezados:} Son metadatos que proporcionan información como el formato de la solicitud, tipo de autenticación
            , entre otros.
    \item \textbf{Datos}: Para los métodos POST y PUT, y otros métodos, la solicitud puede incluir datos.
    \item \textbf{Parámetros:} Brindan al servidor más detalles sobre la solicitud como: filtros, cookies, etc.
\end{itemize}

\subsubsection{Respuesta}
Los principios REST indican que la respuesta del servidor contenga los siguientes componentes principales \cite{aws2025restfulAPI}:

\begin{itemize}
    \item \textbf{Línea de estado:} Contiene un código de estado HTTP de tres dígitos que indica el resultado de la solicitud.
    \item \textbf{Cuerpo del mensaje:} De la mano con los encabezados de la respuesta, contiene la representación del recurso.
    \item \textbf{Encabezados:} Al igual que en la solicitud, estos son metadatos que brindan contexto sobre la respuesta,
            incluyendo información como el servidor, codificación, etc.
\end{itemize}

\subsection{\textit{Performant npm} (PNPM)}
PNPM es un gestor de paquetes para Node JS que se enfoca en ahorro de espacio en disco y mejora la velocidad de instalación
gracias a su enfoque de almacenamiento compartido mediante almacenamiento cnetralizado y enlaces simbólicos, el cuál permite 
que múltiples proyectos compartan dependencias comunes sin la necesidad de duplicar archivos \cite{pnpm2025motivation}.

\subsection{Postman}
Postman es una plataforma de colaboración para el desarrollo, pruebas, documentación y monitoreo de APIs. Es de las más utilizadas
gracias a su interfaz intuitiva y sus múltiples funcionalidades como el envío de solicitudes HTTP/HTTPS con sus respectivos encabezados,
parámetros, cuerpos y autenticación, así como el manejo de las respuestas. Soporta solicitudes RESTful, SOAP, GraphQL, entre otros.
Además, permite la creación de colecciones para organizar las solicitudes y la configuración de entornos para poder gestionar variables
de entorno y saltar de entornos de desarrollo a producción de manera sencilla \cite{postman2025}.
