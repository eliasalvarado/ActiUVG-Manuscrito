La modernización de los procesos institucionales a través de plataformas web se ha convertido en un componente 
esencial para las instituciones de educación superior. La centralización de información y digitalización de trámites 
permiten una mejor experiencia administrativa y académica. Estas herramientas contribuyen a la optimización de gestión
de procesos, fortalece la comunicación entre los actores, y facilita a los estudiantes, directores y organizaciones 
externar interactuar de manera ordenada, transparente y trazable.

El presente proyecto tiene como objetivo desarrollar una plataforma web que centralice la administración, publicación, 
postulación y seguimiento de los proyectos de extensión de la Universidad del Valle de Guatemala. Ateniendo problemas 
actuales de gestión descentralizada: comunicación fragmentada, trámites manuales y baja trazabilidad; limitando la 
participación estudiantil y reduciendo la relevancia de estos proyectos.

La plataforma se estructurará en dos componentes principales: (1) un frontend orientado a directores, organizaciones 
externas y estudiantes para facilitar la interacción y gestión de los proyectos; (2) un backend implementado como una 
interfaz de programación de aplicaciones (API, por sus siglas en inglés) capaz de gestionar usuarios y proyectos. 
Finalmente, estos dos componentes se integrarán utilizando el API desarrollada para comunicar ambas partes. 

Bajo este enfoque, se espera agilizar procesos administrativos vinculados a proyectos de extensión, mejorar la 
trazabilidad y transparencia de las actividades, y aumentar la accesibilidad de las oportunidades para los estudiantes.
