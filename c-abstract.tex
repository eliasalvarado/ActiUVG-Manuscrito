The modernization of institutional processes through web platforms has become an essential component for higher 
education institutions. The centralization of information and digitization of procedures allows for better 
administrative and academic experience. These tools contribute to the optimization of process management, strengthen 
communication between stakeholders, and make it easier for students, directors, and external organizations to interact 
in an orderly, transparent, and traceable manner.

The objective of this project is to develop a web platform that centralizes the administration, publication, 
application, and monitoring of extension projects at the Universidad del Valle de Guatemala. It addresses current 
problems of decentralized management: fragmented communication, manual procedures, and low traceability, which limit 
student participation and reduce the relevance of these projects.

The platform will be structured around two main components: (1) a frontend aimed at directors, external organizations, 
and students to facilitate interaction and project management; (2) a backend implemented as an application programming 
interface (API) capable of managing users and projects. Finally, these two components will be integrated using the API 
developed to communicate between both parties. 

Under this approach, it is expected to streamline administrative processes related to outreach projects, improve the 
traceability and transparency of activities, and increase the accessibility of opportunities for students.
