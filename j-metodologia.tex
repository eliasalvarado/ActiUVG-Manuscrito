En este capítulo se describe la metodología empleada para el desarrollo de la plataforma web,
desde la fase de análisis de requerimientos hasta la unificación de todas las partes del proyecto.

La metodología aplicada se fundamentó en un enfoque estructurado de cinco fases secuenciales e
interrelacionadas, que permitieron una gestión eficiente del proyecto y una integración fluida de
los distintos productos finales de cada fase.

\section{Fase 1: Recolección de Requerimientos}
La primera fase se centró en la extracción de requerimientos técnicos y funcionales a partir del
análisis de los diseños de Interfaz de Usuario previamente elaborador por Andrés Rodríguez en
\cite{Rodríguez_Dis_2023}.

Esta fase tiene como objetivo principal identificar y documentar las reglas de negocio, flujos de
información, roles, restricciones y validaciones necesarias para que la plataforma opere de manera correcta.

\subsection{Definición de módulos funcionales}
Con el fin de identificar acciones, validaciones y flujos de navegación se realizó un análisis detallado
de cada una de las páginas, componentes y formularios presentes en los diseños de interfaz.
Este análisis permitió extraer los módulos funcionales que permitirán un soporte alineado y coherente con 
el fin de la plataforma.
Los módulos funcionales identificados fueron:
\begin{itemize}
    \item \textbf{Módulo de Autenticación:} Este módulo se encarga del registro, inicio de sesión, recuperación
            de contraseña, \textit{logout} y gestión de sesiones de los usuarios. Además, la asignación automática
            de roles a los usuarios.
    \item \textbf{Módulo de Usuarios:} Este módulo se encarga de la actualización de información, cambio de contraseña,
            aprobación de cuentas para usuarios con roles específicos, vista de perfiles. 
    \item \textbf{Módulo de Proyectos:} Este módulo se encarga de la creación, edición, eliminación y visualización de
            proyectos, así como la gestión de sus estados y la aprobación de proyectos por parte de usuarios con roles específicos.
    \item \textbf{Módulo de Postulaciones:} Este módulo se encarga de la gestión de postulaciones a proyectos, incluyendo
            la postulación, revisión y aprobación de postulaciones por parte de usuarios con roles específicos. Además, de
            permitir la visualización de historial de postulaciones.
    \item \textbf{Módulo de Horas:} Este módulo se encarga de la gestión de horas dedicadas a proyectos, incluyendo el registro,
            revisión y aprobación de horas por parte de usuarios con roles específicos. Además, de permitir la visualización de
            historial de horas.
    \item \textbf{Módulo de Archivos:} Este módulo se encarga de la gestión de archivos relacionados con usuarios, proyectos y
            horas, incluyendo la creación y eliminación de archivos, así como la visualización de archivos relacionados a cada entidad.
\end{itemize}

\subsection{Análisis de formularios y mapeo de datos}
Alineado con los módulos funcionales, cada formulario presente en los diseños de interfaz fue desglosado para identificar los
campos de entrada, sus tipos de datos y validaciones necesarias. Permitiendo establecer un mapeo claro entre los datos requeridos
por la plataforma para cada acción posible.

\subsection{Definición de roles y permisos}
A partir de los diseños de interfaz fue sencillo identificar los distintos roles de usuario y sus respectivas acciones permitidas.
De esta manera, se definieron los roles y permisos por usuario, resultando en la siguiente clasificación:
\begin{itemize}
    \item \textbf{Todos los usuarios:}
    \begin{itemize}
        \item Login y registro.
        \item Verificación de correo electrónico.
        \item Restablecimiento de contraseña.
        \item Visualización de proyectos.
        \item Visualización de perfiles de usuario.
        \item Visualización de archivos relacionados a cada entidad.
        \item Visualización de las motivaciones.
        \item Actualización de información personal y cambio de contraseña.
    \end{itemize}
    \item \textbf{Estudiantes:}
    \begin{itemize}
        \item Postulación a proyectos.
        \item Participación en proyectos.
        \item Abandonar proyectos.
        \item Registro de horas trabajadas con evidencias.
        \item Visualización de horas aprobadas en proyectos.
        \item Visualización de historial de postulaciones y horas.
    \end{itemize}
    \item \textbf{Organizaciones externas:}
    \begin{itemize}
        \item Creación, edición y eliminación de proyectos con archivos (recursos) ligados.
        \item Iniciar, finalizar y cancelar proyectos propios.
        \item Visualización de postulaciones a proyectos propios.
        \item Aprobación de postulaciones de proyectos propios.
        \item Visualización de participantes de proyectos propios.
        \item Eliminar participantes de proyectos propios.
        \item Aprobación de horas trabajadas por estudiantes en proyectos propios.
    \end{itemize}
    \item \textbf{Directores:}
    \begin{itemize}
        \item Lo mismo que organizaciones externas.
        \item Aprobación de proyectos enviados por organizaciones externas.
        \item Aprobación de cuentas de organizaciones externas.
    \end{itemize}
\end{itemize}

\subsection{Modelación de la Base de Datos}
Con el fin de garantizar integridad, seguridad y escalabilidad de los datos, se optó por una
base de datos relacional utilizando PostgreSQL como motor de base de datos.
La Figura \ref{fig:ERdb} muestra el diagrama entidad-relación de la base de datos.

\begin{figure}
    \centering
    \includegraphics[width=0.8\textwidth]{assets/ERdb.png}
    \caption{Diagrama Entidad-Relación de la Base de Datos}
    \label{fig:ERdb}
\end{figure}

\subsubsection{Estandarización de estados lógicos}
Con el fin de estandarizar estados lógicos, evitando inconsistencias, se definieron tipos
enumerados (ENUMs) para los roles de los usuarios y estados de proyectos, postulaciones y horas.

\subsubsection{Entidades}
\begin{itemize}
    \item \textbf{Campuses:} Almacena los distintos campus universitarios disponibles.
    \item \textbf{Faculties:} Almacena la información de las facultades académicas.
    \item \textbf{Careers:} Almacena las carreras académicas disponibles.
    \item \textbf{Users:} Almacena la información de los usuarios registrados, principalmente incluyendo información de
            autenticación y rol.
    \item \textbf{Students:} Almacena la información académica y personal de los estudiantes.
    \item \textbf{Directors:} Almacena la información académica de los directores académicos.
    \item \textbf{DirectorFaculties:} Tabla intermedia que permite vincular directores y facultades.
    \item \textbf{DirectorCareers:} Tabla intermedia que permite vincular directores y carreras.
    \item \textbf{Organizations:} Almacena la información de las organizaciones externas.
    \item \textbf{RefreshTokens:} Almacena los tokens de autenticación persistente utilizados para renovar sesiones de usuario.
    \item \textbf{PasswordResetTokens:} Almacena tokens temporales de un solo uso para el proceso de recuperación de contraseña.
    \item \textbf{EmailVerificationTokens:} Almacena los tokens utilizados para verificar la dirección de correo electrónico
            de los usuarios.
    \item \textbf{Projects:} Almacena los proyectos disponibles en la plataforma. Incluyendo información como título, descripción,
            fechas, estado, requisitos, entre otros.
    \item \textbf{ProjectCareers:} Tabla intermedia que permite vincular proyectos con carreras.
    \item \textbf{ProjectTags:} Tabla intermedia que permite vincular etiquetas con proyectos.
    \item \textbf{Applications:} Almacena las postulaciones de los estudiantes a proyectos específicos, incluyendo el estado de
            la aplicación.
    \item \textbf{Hours:} Almacena las horas realizadas por los estudiantes en los proyectos, incluyendo el tipo de hora,
            cantidad y estado de aprobación.
    \item \textbf{Files:} Almacena los archivos asociados al sistema, como fotografías de perfil, recursos de proyectos o
            evidencias de horas.
\end{itemize}

\subsubsection{Relaciones}
\begin{itemize}
    \item Un campus puede tener múltiples facultades, pero cada facultad pertenece a un único campus.
    \item Una facultad puede tener varias carreras, pero cada carrera pertenece a una sola facultad.
    \item Un usuario puede estar asociado a un único estudiante, director u organización.
    \item Un director puede tener varias facultades.
    \item Un director puede tener varias carreras.
    \item Una organización puede crear múltiples proyectos, pero cada proyecto pertenece a una única organización (cuando aplica).
    \item Un director puede crear múltiples proyectos, pero cada proyecto pertenece a un único director (cuando aplica).
    \item Un proyecto puede estar disponible para varias carreras y una carrera puede participar en múltiples proyectos.
    \item Un proyecto puede recibir múltiples postulaciones, pero cada postulación corresponde a un único proyecto.
    \item Un estudiante puede postularse a múltiples proyectos, pero cada postulación pertenece a un único estudiante.
    \item Un estudiante puede registrar múltiples transacciones de horas en distintos proyectos.
    \item Un proyecto puede tener múltiples registros de horas asociados.
    \item Un usuario puede subir múltiples archivos.
    \item Un proyecto puede tener múltiples archivos asociados como recursos.
    \item Un registro de horas puede tener múltiples archivos asociados como evidencias.
    \item Un usuario puede tener múltiples tokens de autenticación, verificación o recuperación asociados.
\end{itemize}

\subsection{Definición de \textit{Endpoints}}
A partir de los módulos funcionales y el análisis de formularios se estructuró la arquitectura \textit{RESTful} de la API.
Definiendo contratos de comunicación claros, incluyendo rutas, verbos HTTP, parámetros de entrada, formatos de respuesta
y códigos de estado.

\subsubsection{Autenticación}
Ruta base: \texttt{/auth}. Este conjunto de \textit{endpoints} se encarga de la autenticación, recuperación de contraseña,
verificación de correo electrónico, gestión de sesiones y tokens de autenticación.
\begin{itemize}
    \item \texttt{POST /auth/register}: Registro de usuarios.
    \item \texttt{POST /auth/login}: Inicio de sesión.
    \item \texttt{POST /auth/refresh}: Renovación de tokens de autenticación.
    \item \texttt{POST /auth/verify-email}: Verificación de correo electrónico.
    \item \texttt{POST /auth/forgot-password}: Solicitud de restablecimiento de contraseña.
    \item \texttt{POST /auth/reset-password}: Confirmación de restablecimiento de contraseña con token.
    \item \texttt{POST /auth/logout}: Cierre de una sesión específica.
    \item \texttt{POST /auth/logout-all}: Cierre de todas las sesiones activas.
    \item \texttt{GET /auth/sessions}: Listado de las sesiones activas.
\end{itemize}

\subsubsection{Usuarios}
Ruta base: \texttt{/users}. Este conjunto de \textit{endpoints} gestiona la información de los usuarios registrados,
incluyendo consulta de perfiles, actualización de datos, cambio de contraseña y eliminación de cuentas.
\begin{itemize}
    \item \texttt{GET /users/}: Listar estudiantes.
    \item \texttt{GET /users/me}: Obtener perfil propio.
    \item \texttt{GET /users/:id/profile}: Obtener perfil público de un usuario.
    \item \texttt{GET /users/:userId}: Obtener usuario por ID (administrador).
    \item \texttt{PUT /users/:id}: Actualizar información de usuario.
    \item \texttt{PUT /users/:id/change-password}: Cambiar contraseña.
    \item \texttt{DELETE /users/:id}: Eliminar usuario.
\end{itemize}

\subsubsection{Proyectos}
Ruta base: \texttt{/projects}. Este conjunto de \textit{endpoints} gestiona la creación, consulta, actualización,
aprobación y eliminación de proyectos.
\begin{itemize}
    \item \texttt{GET /projects}: Listar proyectos con filtros.
    \item \texttt{GET /projects/my-projects}: Listar proyectos propios.
    \item \texttt{GET /projects/pending-approvals}: Listar proyectos pendientes de aprobación o rechazo.
    \item \texttt{GET /projects/:projectId}: Obtener proyecto por ID.
    \item \texttt{POST /projects}: Crear proyecto.
    \item \texttt{PUT /projects/:projectId}: Actualizar proyecto.
    \item \texttt{POST /projects/:projectId/approve}: Aprobar o rechazar proyecto.
    \item \texttt{PATCH /projects/:projectId/status}: Cambiar estado del proyecto.
    \item \texttt{DELETE /projects/:projectId}: Eliminar proyecto.
\end{itemize}

\subsubsection{Aplicaciones}
Ruta base: \texttt{/applications}. Este conjunto de \textit{endpoints} gestiona las postulaciones de estudiantes
a proyectos y la administración de participantes.
\begin{itemize}
    \item \texttt{POST /applications}: Aplicar a un proyecto.
    \item \texttt{GET /applications/completed}: Obtener historial de proyectos finalizados.
    \item \texttt{GET /applications/project/:projectId}: Obtener aplicaciones de un proyecto.
    \item \texttt{GET /applications/project/:projectId/participants}: Obtener participantes aceptados de un proyecto.
    \item \texttt{PATCH /applications/:applicationId/status}: Aprobar o rechazar aplicación.
    \item \texttt{PATCH /applications/project/:projectId/leave}: Abandonar proyecto.
    \item \texttt{PATCH /applications/project/:projectId/participant/:studentId}: Eliminar participante de un proyecto.
\end{itemize}

\subsubsection{Horas}
Ruta base: \texttt{/hours}. Este conjunto de \textit{endpoints} gestiona el registro, consulta, aprobación
y reenvío de horas asociadas a proyectos.
\begin{itemize}
    \item \texttt{POST /hours}: Registrar horas con evidencias.
    \item \texttt{GET /hours/total}: Obtener total de horas aprobadas.
    \item \texttt{GET /hours/:hourId}: Obtener detalles de un registro de horas.
    \item \texttt{GET /hours/project/:projectId/pending}: Obtener horas pendientes de aprobación en un proyecto.
    \item \texttt{PATCH /hours/:hourId/status}: Aprobar o rechazar horas.
    \item \texttt{POST /hours/:hourId/evidences}: Agregar evidencias a registro.
    \item \texttt{DELETE /hours/:hourId/evidences/:fileId}: Eliminar evidencia específica.
    \item \texttt{PATCH /hours/:hourId/resubmit}: Reenviar horas rechazadas.
\end{itemize}

\subsubsection{Archivos}
Ruta base: \texttt{/files}. Este conjunto de \textit{endpoints} gestiona la descarga o previsualización de los archivos
de la plataforma. Esto incluye fotos de perfil, recursos de proyectos y evidencias de horas.
\begin{itemize}
    \item \texttt{GET /files/:id/download}: Descargar archivo.
    \item \texttt{GET /files/:id/preview}: Previsualizar archivo.
\end{itemize}
