En este capítulo se describe la metodología empleada para el desarrollo de la plataforma web,
desde la fase de análisis de requerimientos hasta la unificación de todas las partes del proyecto.

La metodología aplicada se fundamentó en un enfoque estructurado de cinco fases secuenciales e
interrelacionadas, que permitieron una gestión eficiente del proyecto y una integración fluida de
los distintos productos finales de cada fase.

\section{Fase 1: Recolección de Requerimientos}
La primera fase se centró en la extracción de requerimientos técnicos y funcionales a partir del
análisis de los diseños de Interfaz de Usuario previamente elaborador por Andrés Rodríguez en
\cite{Rodríguez_Dis_2023}.

Esta fase tiene como objetivo principal identificar y documentar las reglas de negocio, flujos de
información, roles, restricciones y validaciones necesarias para que la plataforma opere de manera correcta.

\subsection{Definición de módulos funcionales}
Con el fin de identificar acciones, validaciones y flujos de navegación se realizó un análisis detallado
de cada una de las páginas, componentes y formularios presentes en los diseños de interfaz.
Este análisis permitió extraer los módulos funcionales que permitirán un soporte alineado y coherente con 
el fin de la plataforma.
Los módulos funcionales identificados fueron:
\begin{itemize}
    \item \textbf{Módulo de Autenticación:} Este módulo se encarga del registro, inicio de sesión, recuperación
            de contraseña, \textit{logout} y gestión de sesiones de los usuarios. Además, la asignación automática
            de roles a los usuarios.
    \item \textbf{Módulo de Usuarios:} Este módulo se encarga de la actualización de información, cambio de contraseña,
            aprobación de cuentas para usuarios con roles específicos, vista de perfiles. 
    \item \textbf{Módulo de Proyectos:} Este módulo se encarga de la creación, edición, eliminación y visualización de
            proyectos, así como la gestión de sus estados y la aprobación de proyectos por parte de usuarios con roles específicos.
    \item \textbf{Módulo de Postulaciones:} Este módulo se encarga de la gestión de postulaciones a proyectos, incluyendo
            la postulación, revisión y aprobación de postulaciones por parte de usuarios con roles específicos. Además, de
            permitir la visualización de historial de postulaciones.
    \item \textbf{Módulo de Horas:} Este módulo se encarga de la gestión de horas dedicadas a proyectos, incluyendo el registro,
            revisión y aprobación de horas por parte de usuarios con roles específicos. Además, de permitir la visualización de
            historial de horas.
    \item \textbf{Módulo de Archivos:} Este módulo se encarga de la gestión de archivos relacionados con usuarios, proyectos y
            horas, incluyendo la creación y eliminación de archivos, así como la visualización de archivos relacionados a cada entidad.
\end{itemize}

\subsection{Análisis de formularios y mapeo de datos}
Alineado con los módulos funcionales, cada formulario presente en los diseños de interfaz fue desglosado para identificar los
campos de entrada, sus tipos de datos y validaciones necesarias. Permitiendo establecer un mapeo claro entre los datos requeridos
por la plataforma para cada acción posible.

\subsection{Definición de roles y permisos}
A partir de los diseños de interfaz fue sencillo identificar los distintos roles de usuario y sus respectivas acciones permitidas.
De esta manera, se definieron los roles y permisos por usuario, resultando en la siguiente clasificación:
\begin{itemize}
    \item \textbf{Todos los usuarios:}
    \begin{itemize}
        \item Login y registro.
        \item Verificación de correo electrónico.
        \item Restablecimiento de contraseña.
        \item Visualización de proyectos.
        \item Visualización de perfiles de usuario.
        \item Visualización de archivos relacionados a cada entidad.
        \item Visualización de historial de postulaciones y horas.
        \item Visualización de las motivaciones.
        \item Actualización de información personal y cambio de contraseña.
    \end{itemize}
    \item \textbf{Estudiantes:}
    \begin{itemize}
        \item Postulación a proyectos.
        \item Participación en proyectos.
        \item Abandonar proyectos.
        \item Registro de horas trabajadas con evidencias.
        \item Visualización de horas aprobadas en proyectos.
    \end{itemize}
    \item \textbf{Organizaciones externas:}
    \begin{itemize}
        \item Creación, edición y eliminación de proyectos con archivos (recursos) ligados.
        \item Iniciar, finalizar y cancelar proyectos propios.
        \item Visualización de postulaciones a proyectos propios.
        \item Aprobación de postulaciones.
        \item Visualización de participantes de proyectos propios.
        \item Eliminar participantes de proyectos propios.
        \item Aprobación de horas trabajadas por estudiantes en proyectos propios.
    \end{itemize}
    \item \textbf{Directores:}
    \begin{itemize}
        \item Lo mismo que organizaciones externas.
        \item Aprobación de proyectos enviados por organizaciones externas.
        \item Aprobación de cuentas de organizaciones externas.
    \end{itemize}
\end{itemize}

\subsection{Modelación de la Base de Datos}:
