\newglossaryentry{latex}
{
    name=latex,
    description={Es un lenguaje de marcado adecuado especialmente para la creación de documentos científicos}
} 
 
\newglossaryentry{formula}
{
    name=fórmula,
    description={Una expresión matemática} 
}

\newglossaryentry{frontend}
{
    name=Frontend,
    description={Capa de la aplicación que interactúa directamente con el usuario, abarcando la interfaz gráfica y la experiencia de navegación.}
}

\newglossaryentry{renderizado}
{
    name=Renderizado Web,
    description={Proceso informático mediante el cual un motor (como un navegador web) interpreta código (HTML, CSS, JS) para mostrar una interfaz gráfica interactiva al usuario.}
}

\newglossaryentry{backend}
{
    name=Backend,
    description={Capa de la aplicación encargada de la lógica de negocio, procesamiento, seguridad y comunicación con la base de datos.}
}

\newglossaryentry{cloudcomputing}{
    name=Cloud Computing (Computación en la Nube),
    description={Modelo de computación que permite el acceso a recursos informáticos (servidores, almacenamiento, bases de datos, redes, software) a través de Internet, ofreciendo escalabilidad, flexibilidad y eficiencia.}
}

\newglossaryentry{container}
{
    name=Contenedor,
    description={Unidad de software ligera y portátil que empaqueta código y dependencias para ejecutarse de manera uniforme en cualquier entorno.}
}

\newglossaryentry{type-safe}
{
    name=Type-safe (Tipado Seguro),
    description={Característica de un lenguaje de programación (como TypeScript) que previene errores al obligar a definir y respetar los tipos de datos en tiempo de desarrollo.}
}

\newglossaryentry{isomorfico}
{
    name=Isomórfico (Código),
    description={Código que tiene la capacidad de ejecutarse tanto en el entorno del cliente (navegador) como en el entorno del servidor (ej. Node.js).}
}

\newglossaryentry{endpoint}
{
    name=Endpoint,
    description={Punto de acceso de comunicación (generalmente una URL) mediante el cual un cliente interactúa con una API.}
}

\newglossaryentry{hashing}
{
    name=Hashing,
    description={Proceso de transformar datos de entrada (como una contraseña) en una cadena de caracteres de longitud fija, utilizando un algoritmo específico, con el fin de proteger la información.}
}

\newglossaryentry{log}
{
    name=Log,
    description={Registro de eventos o actividades que ocurren dentro de un sistema, utilizado para monitoreo, depuración y análisis.}
}

\newglossaryentry{script}
{
    name=Script,
    description={Archivo de texto que contiene código ejecutable, generalmente utilizado para automatizar tareas o ejecutar funciones específicas dentro de un programa o sistema.}
}