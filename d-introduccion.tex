La extensión universitaria tiene un papel fundamental en el desarrollo estudiantil. Genera un vínculo entre las 
instituciones de educación superior y la sociedad, ya que permite que los estudiantes y organizaciones externas 
colaboren en actividades orientadas al desarrollo comunitario y formación integral \cite{unu2025higherEducation}; 
sin embargo, muchas veces la gestión de estos procesos se realiza de manera descentralizada, comunicación dispersa 
y procedimientos manuales los cuales dificultan la organización, difusión y seguimiento de los proyectos disponibles 
\cite{worldbank2009challenge}. Debido a estas limitaciones, la participación estudiantil se ve afectada, ya que se 
reduce la eficiencia administrativa y se genera una experiencia mal estructurada para todos los actores involucrados.

La modernización mediante plataformas digitales ha impulsado nuevas oportunidades y diversos estudios destacan 
que los sistemas centralizados mejoran la accesibilidad de la información, permiten un mejor control sobre los 
procesos incluidos y, sobre todo, facilitan la toma de decisiones \cite{MCCARTHY2023100479}. En el entorno universitario, la digitalización 
de flujos de distintos procesos académicos y administrativos es esencial para aumentar la productividad \cite{edutech2025digitalCampus}. 
Actualmente, la Universidad del Valle de Guatemala realiza de forma manual las distintas gestiones para motivar al 
estudiante a participar en proyectos de extensión universitaria; esto conlleva la difusión inicial de la propuesta 
del proyecto por parte del director de carrera a través de correo electrónico, aceptación de los estudiantes 
postulados y validación del trabajo realizado por el estudiante. Terminando en un proceso poco efectivo, trabajoso 
y, en muchas ocasiones, poco trazable de inicio a fin \cite{reglamento2022uvg}. Afectado de gran manera a los estudiantes, ya que la 
mayoría de las ocasiones, el estudiante no llega a participar en muchos proyectos al ignorar por completo su 
existencia.

En base a esta necesidad, este proyecto propone desarrollar una herramienta web que unifique el ciclo completo de 
los proyectos de extensión: creación por parte de la organización externa, aprobación por parte del director de 
Carrera, publicación del proyecto de extensión universitaria, postulación de los estudiantes, seguimiento y 
validación final por parte del director de Carrera y la organización externa. Una plataforma web integrada no 
solo permitiría agilizar las gestiones internas, sino que también ofrecer una experiencia más agradable y accesible 
para los estudiantes, quienes tendrán la información centralizada para poder participar en actividades formativas \cite{PINHO201880}.

Este proyecto busca contribuir a la modernización institucional y fortalecer los distintos mecanismos de 
interacción entre la Universidad, su comunidad y organizaciones externas \cite{urmeneta2025design}.
